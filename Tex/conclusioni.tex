\chapter*{Conclusioni}
\label{ch:conclusioni}

\begin{flushleft}
%\section{Conclusioni}
In conclusione della mia tesi, desidero esprimere la mia soddisfazione per i risultati ottenuti e l'opportunità di immergermi nel mondo del machine learning. Questa esperienza è stata fondamentale per la mia formazione, consentendomi di acquisire conoscenza e mettere in pratica le competenze nel campo dell'intelligenza artificiale.

Durante il mio lavoro ho sviluppato 3 modelli di ML diversi.  In tutti e tre i casi, il Random Forest ha dimostrato di essere il modello più performante e stabile.
Ho avuto modo di lavorare sia con parametri clinici ( tra cui meritano menzione "ST slope","ST depression" e "chest pain type") sia con parametri personali.
Su questi è stato per me importante menzionare l'impatto che hanno avuto "Smoking" e "Sleep Time", tali risultati suggeriscono che migliorare il proprio stato di salute può anche dipendere dallo stile di vita che si adotta quotidianamente.

La possibilità di studiare e applicare metodi di XAI è stato importante per una maggiore comprensione delle dinamiche interne ai modelli, sia per me durante la costruzione,che per i clinici che  otterranno giovamento dalla maggiore chiarezza
e dalle possibilità che offre in ambito di riabilitazione.

Guardando verso il futuro, il mio tool di esplorazione dei dati è ancora in fase di sviluppo. Tuttavia, io e il team a cui è destinato siamo fiduciosi nel suo potenziale e nei possibili sviluppi che potrà offrire. L'obiettivo principale è di rendere il tool ancora più completo nel breve termine, in attesa dei dati clinici da parte degli ospedali, su cui costruire i nuovi modelli.

Un'area di interesse che non è stata menzionata durante la tesi è l'imputazione dei dati per la creazione di un dataset unico clinico, risultato del merging degli altri dataset trovati, sulla base delle loro feature comuni. 
Ho condotto dei test preliminari in questa direzione, esplorando metodi di imputazione per gestire i dati mancanti e garantire la coerenza e l'affidabilità del dataset. 
Tuttavia i risultati non hanno suggerito la percorribilità di questo percorso.

In conclusione, sono entusiasta dei risultati ottenuti e sono grato per l'opportunità di immergermi nel campo del machine learning. Guardando avanti, sono estremamente contento ed entusiasta che il mio lavoro abbia contribuito, seppur in minima parte, ad un ausilio per la risoluzione di un tale problema di classificazione, e sono fiducioso che il mio lavoro possa avere un impatto significativo, sia internamente che esternamente a Dedalus, sulla vita delle altre persone.

\end{flushleft}
